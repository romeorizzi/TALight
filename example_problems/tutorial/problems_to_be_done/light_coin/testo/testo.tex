% Template per generare

\documentclass[a4paper,11pt]{article}
\usepackage{lmodern}
\renewcommand*\familydefault{\sfdefault}
\usepackage{sfmath}
\usepackage[utf8]{inputenc}
\usepackage[T1]{fontenc}
\usepackage[italian]{babel}
\usepackage{indentfirst}
\usepackage{graphicx}
\usepackage{tikz}
\newcommand*\circled[1]{\tikz[baseline=(char.base)]{
    \node[shape=circle,draw,inner sep=2pt] (char) {#1};}}
\usepackage{enumitem}
% \usepackage[group-separator={\,}]{siunitx}
\usepackage[left=2cm, right=2cm, bottom=3cm]{geometry}
\frenchspacing

\newcommand{\num}[1]{#1}

% Macro varie...
\newcommand{\file}[1]{\texttt{#1}}
\renewcommand{\arraystretch}{1.3}
\newcommand{\esempio}[2]{
  \noindent\begin{minipage}{\textwidth}
    \begin{tabular}{|p{11cm}|p{5cm}|}
      \hline
      \textbf{File \file{input.txt}} & \textbf{File \file{output.txt}}\\
      \hline
      \tt \small #1 &
      \tt \small #2 \\
      \hline
    \end{tabular}
  \end{minipage}
}

% Dati del task
\newcommand{\gara}{.}
\newcommand{\nome}{Individuare la Moneta Leggera}
\newcommand{\nomebreve}{light\_coin}

\begin{document}
  
  
  % Intestazione
  \noindent{\Large \gara}
  \vspace{0.5cm}
  
  \noindent{\Huge \textbf \nome~(\texttt{\nomebreve})}
  \vspace{0.2cm}\\
  
  % Descrizione del task
  \section*{Descrizione del problema}
    
  \noindent
  Devi scrivere una procedura che individui l'unica
  moneta falsa in un set di $n$ monete numerate da $0$ ad $n-1$.
  Il file da sottomettere deve avere la seguente struttura:
\begin{verbatim}
#include "ourLibToPlay.h"

void individua(long int n) {
   ...
} 
\end{verbatim}

  \noindent
  Il parametro $n$ che viene passato alla funzione \texttt{individua} 
  \`e il numero di monete sotto esame. Tutte le $n$ monete hanno lo stesso peso,
  tranne quella falsa, che \`e pi\`u leggera delle altre.
  Potrai servirti di una bilancia a braccia eguali
  invocando, dalla tua implementazione della procedura \texttt{individua},
  la seguente funzione:
  
  \vspace{0.2cm}
  
  \noindent
  \texttt{int piatto\_con\_peso\_maggiore()}
  
  \vspace{0.2cm}
  
  \noindent
  La funzione prevede i seguenti $3$ possibili valori di ritorno:
\[
    \mbox{\texttt{piatto\_con\_peso\_maggiore()}} \, =
    \left\{
       \begin{array}{ll}
          \texttt{NONE} & \mbox{se i due piatti della bilancia sono in perfetto equilibrio;}\\
          \texttt{LEFT} & \mbox{se il carico \`e maggiore sul piatto sinistro;}\\
          \texttt{RIGHT} & \mbox{se il carico \`e maggiore sul piatto destro.}\\
       \end{array}
    \right.
\]
  dove \texttt{LEFT}$=-1$, \texttt{NONE}$=0$ e \texttt{RIGHT}$=1$
  sono $3$ costanti intere definite per voi in \texttt{ourLibToPlay.h}. 
  
  \noindent
  Per portare una certa moneta da dove si trova attualmente ad un certo piatto
  (\texttt{LEFT}, \texttt{RIGHT}, oppure anche \texttt{NONE} nel caso si voglia
   togliere la moneta dalla bilancia)
   si invoca la procedura:

  \vspace{0.2cm}

  \noindent
  \texttt{void collocaMoneta(long int moneta, int piatto)}

  \vspace{0.2cm}

  \noindent
  Quando trova la moneta falsa, la tua procedura deve consegnarla
  alla zecca invocando:
  
  \vspace{0.2cm}
  
  \noindent
  \texttt{void denuncia(long int monetaFalsa)}
  
  
  \section*{Subtask}
  \begin{itemize}
    \item \textbf{Subtask 0 [2 punti]:} la moneta falsa è la $2$.
    \item \textbf{Subtask 1 [3 punti]:} la moneta falsa è la $0$ oppure la $1$.
    \item \textbf{Subtask 2 [5 punti]:} $n=7$ e sono consentite al pi\`u $6$ pesate.
    \item \textbf{Subtask 3 [5 punti]:} $n=7$ e sono consentite al pi\`u $4$ pesate.
    \item \textbf{Subtask 4 [10 punti]:} $n=7$ e sono consentite al pi\`u $3$ pesate.
    \item \textbf{Subtask 5 [10 punti]:} $n=8$ e sono consentite al pi\`u $3$ pesate.
    \item \textbf{Subtask 6 [10 punti]:} vengono consentite al pi\`u $n-1$ pesate.
    \item \textbf{Subtask 7 [10 punti]:} vengono consentite al pi\`u $\lfloor n/2 \rfloor$ pesate.
    \item \textbf{Subtask 8 [25 punti]:} vengono consentite al pi\`u $\lceil \log_2 n \rceil$ pesate.
    \item \textbf{Subtask 9 [25 punti]:} viene permesso solo quel minimo numero di pesate che, se impiegato sapientemente, consenta sempre di individuare la moneta falsa.
  \end{itemize}
  
  % Assunzioni
  \section*{Assunzioni}
  \begin{itemize}[nolistsep, noitemsep]
    \item Il programma termina dopo la prima chiamata alla funzione \texttt{denuncia} oppure allo scadere del tempo limite.
    \item $1 \le n \le 1\,000\,000$.
  \end{itemize}


  % Cosa sottomettere
  \section*{Cosa deve contenere il File da Sottomettere, e come viene gestito dal server}

  L'estensione del file che sottometti, *.c, *.cpp oppure *.pas,
  determina se esso viene compilato col compilatore del C (gcc),
  del c++ (g++) oppure del PASCAL (fpc).
  Le opzioni di compilazione, specifiche al linguaggio,
  sono riportate nel dettaglio sulla pagina "Descrizione" del problema
  in modo che tu possa simulare esattamente in locale cosa succede sul server.
  I comandi di compilazione riportati assumono che il nome del file sottomesso (privato di estensione) coincida col nome del problema.

  Nel caso di un problema interattivo come questo,
  dove il tuo programma interagisce con altri da noi predisposti
  (il grader ed anche un header da includere), i comandi di compilazione riportati compilano ed assemblano i vari pezzi (il tuo programma, il grader, e l'header).
  Se vuoi testare il comportamento del tutto in locale,
  devi allora scaricarti il nostro grader e l'header che rendiamo disponibile tra gli allegati alla pagina del problema.

  Tra gli allegati potremmo inoltre fornire una soluzione di esempio che,
  pur essendo compilabile sia in locale che dal server (ove ad esso sottomessa),
  faccia pochi o nessun punto in quanto non significativa per il problema
  in questione.
  In questo caso l'esempio compilabile reso disponibile al sito \`e il seguente:
\begin{verbatim}
//problem: lightCoin, example of a solution file, Romeo Rizzi Jan 2015
#include "ourLibToPlay.h"
void individua(long int n) {
  long int i;
  for(i = 0; i<n; i++) {
      if(i%3 == 1) collocaMoneta(i, LEFT);
      if(i%3 == 2) collocaMoneta(i, RIGHT);
  }
  int risp =  piatto_con_peso_maggiore();
  for(i = 0; i<n; i++)  collocaMoneta(i, NONE);
  if(risp == NONE) denuncia(0); //potrebbe essere la moneta 0 (sul piatto NONE)
  if(risp == LEFT) denuncia(2); //potrebbe essere la moneta 2 (sul piatto RIGHT) 
  if(risp == RIGHT) denuncia(1); //potrebbe essere la moneta 1 (sul piatto LEFT)
}
\end{verbatim}

Non risolver\`a molte istanze ma \`e compilabile sia in locale (se nella stessa cartella collocate anche il grader e l'header resi disponibili alla pagina del problema) sia ove sottoposto al server.
  
  
\end{document}
